\begin{task}{442}
Пусть дан двудольный граф (необязательно планарный) \(G\) на \(100\) вершинах, в котором минимальная длина цикла равна \(6\). Используя технику двойного подсчета и неравенство между средним квадратическим и средним арифметическим, докажите, что \(\|G\| < 400\). \textit{Подсказка}: посчитайте двумя способами для каждой из долей число конфигураций вида \((u,v,w)\), где \(v\)~--- вершина данной доли, а \(u,w\)~--- некоторые её соседи.
\end{task}

\begin{solution}
Обозначим доли как $M$ и $N$, а количество вершин в них -- через $m$ и $n$ соответственно. Количество рёбер будем обозначать через $e$, число конфигураций вида $(u,v,w)$ (где $v$ -- вершина доли $M$, а $u,w$ -- некоторые её соседи) --- через $c$. Из ограничения на минимальную длину цикла заключаем, что в исходном графе нет подграфа, изоморфного $K_{2,2}$. Учитывая
\begin{equation}\label{eq0}
    e=\sum_{v\in M}{\deg{v}}
\end{equation}
(из двудольности), число конфигураций \begin{equation}\label{eq1}
c \leq C_{n}^{2}.
\end{equation} С другой стороны, $c = \sum\limits_{v\in M} C_{\deg{v}}^{2}.$ Подставив последнее выражение в качестве $c$ в~\eqref{eq1}, расписав биномиальный коэффициент и умножив обе части неравенства на $2$, получим
\begin{equation}\label{eq2}
\sum_{v\in M} {\deg^2{v}} - e \leq n(n-1).
\end{equation}
Из~\eqref{eq0} и неравенства между средним квадратическим и средним арифметическим следует $\frac{e^2}{m} \leq \sum\limits_{v\in M}{\deg^2{v}}$. Тогда из~\eqref{eq2} получаем
\[ e^2 \leq em + mn(n-1).\]
Для другой доли такими же рассуждениями приходим к
\[e^2 \leq en + mn(m-1).\]
Сложим два полученных неравенства:
\begin{gather*}
2e^2 \leq e(m + n) + mn(m + n - 2) \\
2e^2 - e(m + n) - mn(m + n - 2) \leq 0 \\
e^2 - 50e - 49mn \leq 0.
\end{gather*}
Учитывая $mn \leq 2500$ (сумма $m$ и $n$ равна $100$, а функция $f=x(100-x)$ имеет максимум в точке $(50, 50)$), находим
\[ e \leq \frac{50 + \sqrt{50 + 4 \cdot 49 \cdot 2500}}{2} \approx 375. \]
Таким образом, $e < 400$, что и требовалось доказать.
\end{solution}
